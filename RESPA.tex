\documentclass[
    journal=jctcce,
    layout=twocolumn
]{achemso}

\setkeys{acs}{
	abbreviations = false,
	articletitle  = false,
	keywords      = false,
	maxauthors    = 10,
	super         = true
}

% Comment below before submitting:
\let\titlefont\undefined
\usepackage[fontsize=11pt]{scrextend}
\usepackage[hidelinks,colorlinks,citecolor=blue]{hyperref}
%\flushbottom
% Up to this point

\usepackage{amsmath}
\usepackage{amssymb}
\usepackage[T1]{fontenc}
%
\usepackage[table]{xcolor}
\definecolor{lightgray}{gray}{0.85}
%
\usepackage{array}
\newcolumntype{L}{>{$}l<{$}}
\newcolumntype{C}{>{$}c<{$}}
\newcolumntype{R}{>{$}r<{$}}
%
\newcommand{\mt}[1]{\boldsymbol{\mathbf{#1}}}   % matrix symbol
\newcommand{\vt}[1]{\boldsymbol{\mathbf{#1}}}   % vector symbol
\newcommand{\tr}[1]{#1^\text{t}}                % transposition
\newcommand{\diff}[2]{\frac{\partial #2}{\partial #1}} % derivative
%\newcommand{\diff}[2]{\nabla_{#1}{#2}} % derivative
%\newcommand{\diff}[2]{\partial_{#1}{#2}} % derivative
\newcommand{\avg}[1]{\overline{#1}}             % average

\newcommand{\dof}{i}   % index for each degree of freedom

%\listfiles

\author{Charlles R. A. Abreu}
\email{abreu@eq.ufrj.br}
\affiliation{Chemical Engineering Department, Escola de Quimica, Universidade Federal do Rio de Janeiro, Rio de Janeiro, RJ 21941-909, Brazil}
\alsoaffiliation{Department of Chemistry, New York University, New York, New York 10003, USA}

\author{Mark E. Tuckerman}
\email{marktuckerman@nyu.edu}
\affiliation{Department of Chemistry, New York University, New York, New York 10003, USA}
\alsoaffiliation{Courant Institute of Mathematical Sciences, New York University, New York, New York 10012, USA}
\alsoaffiliation{NYU-ECNU Center for Computational Chemistry at NYU Shanghai, Shanghai 200062, China}


\title{Solvation Free-Energy Calculation via Molecular Dynamics with Large Time Steps}

\abbreviations{i.i.d., MC, MD, CLT, OBM, MSE, FEP, BAR, WHAM, MBAR, MICS}

\keywords{Free Energy Computation, Reweighting, Multistate, Uncertainty Estimation}

\begin{document}

%\begin{tocentry}
%Graphical Abstract
%\end{tocentry}

%\tableofcontents

\begin{abstract}
To be included.
\end{abstract}

\section{Introduction}
\label{sec:introduction}

The SIN(R) method \cite{Leimkuhler_2013}.

Isokinetic distributions coincides with a canonical distribution if we look at the coordinates only.

\section{Method}

\subsection{Equations of Motion}

Consider a system of $N$ particles in $d$ dimensions, whose each configurational degree of freedom (DoF) $\dof$ has coordinate $q_\dof$, velocity $v_\dof$, and associated mass $m_\dof$.
The system is subject to a potential $U(\vt q)$ and, therefore, the force acting on $\dof$ is $F_\dof = -\diff{q_\dof}{U}$.
In the SIN(R) method, every DoF gets acted upon by an individual set of thermostats, which is known as massive thermostatting.
For this, we define a pair of extended-system variables $v_{1,\dof}$ and $v_{2,\dof}$ for every DoF $\dof$, with associated inertial parameters $Q_1$ and $Q_2$, respectively.
In practice, a single value $Q_1 = Q_2 = kT\tau^2$ is employed, where $k$ is the Boltzmann constant, $T$ is the temperature of the heat bath, and $\tau$ is a relevant time scale of the system dynamics.
The stochastic differential equation (SDE) system which prescribes such dynamics is
\begin{subequations}
\label{eq:equations of motion}
\begin{align}
& \dot{q}_\dof = v_\dof, \label{eq:q} \\
& \dot{v}_\dof = \frac{F_\dof}{m_\dof} - \lambda_\dof v_\dof, \label{eq:v} \\
& \dot{v}_{1,\dof} = - \lambda_\dof v_{1,\dof} - v_{2,\dof} v_{1,\dof}, \quad \mathrm{and}  \label{eq:v1} \\
& dv_{2,\dof} = \tfrac{Q_1 v_{1,\dof}^2 - kT}{Q_2}dt - \gamma v_{2,\dof} dt + \sqrt{\tfrac{2 \gamma kT}{Q_2}} dW_\dof, \label{eq:v2}
\end{align}
\end{subequations}
where $\gamma$ is a friction constant and $dW_\dof$ represents an infinitesimal increment of a stochastic Wiener process.
In addition, a Lagrange multiplier $\lambda_\dof$ is introduced with the aim of imposing an isokinetic constraint to each DoF, which is \cite{Leimkuhler_2013, Margul_2016}
\begin{subequations}
\label{eq:isokinetic constraint}
\begin{equation}
m v_\dof^2 + \frac{1}{2} Q_1 v_{1,\dof}^2 = kT.
\end{equation}

Since this implies that
\begin{equation}
\label{eq:isokinetic constraint derivative}
m v_\dof \dot{v}_\dof + \frac{1}{2} Q_1 v_{1,\dof}\dot{v}_{1,\dof} = 0,
\end{equation}
\end{subequations}
substitution of $\dot{v}_\dof$ and $\dot{v}_{1,\dof}$ from Eqs.~\eqref{eq:v} and \eqref{eq:v1} leads to
\begin{equation}
\lambda_\dof = \frac{F_\dof v_\dof - \frac{1}{2} Q_1 v_{2,\dof} v_{1,\dof}^2}{m_\dof v_\dof^2 + \frac{1}{2} Q_1 v_{1,\dof}^2}.
\end{equation}

As usual, we can express a set of equations of motion as the effect of a Liouville operator over the dynamic variables of the system.
In the present case, such operator can be expressed as
\begin{multline}
\label{eq:isokinetic Liouville operator}
iL = \sum_{\dof=1}^{dN} \bigg[ v_\dof\diff{q_\dof}{} + \left(\frac{F_\dof}{m_\dof} - \lambda_\dof v_\dof\right)\diff{v_\dof}{} \\
- \left( \lambda_\dof v_{1,\dof} + v_{2,\dof} v_{1,\dof} \right) \diff{v_{1,\dof}}{} \\
+ \tfrac{Q_1 v_{1,\dof}^2 - kT}{Q_2}\diff{v_{2,\dof}}{}
+ iL_{OU,\dof} \bigg],
\end{multline}
where $iL_{OU,\dof}$ would, if isolated, represent a drift-free Ornstein-Uhlenbeck process involving variable $v_{2,\dof}$.
As demonstrated in Ref.~\citenum{Leimkuhler_2013}, the SIN(R) method dynamics is ergodic and preserves the isokinetic distribution.
The equations resemble a Nos\'{e}-Hoover-Langevin thermostat \cite{Samoletov_2007, Leimkuhler_2009}, with $v_{1,\dof}$ acting as intermediary between the system DoF $\dof$ and a Langevin-type thermostat.
It is also possible to consider the simultaneous action of a set of $2L$ thermostat variables per DoF \cite{Minary_2003, Minary_2003_2}.
Throughout this document, however, we only adopt the simplest case ($L=1$) because the method's efficacy has been verified to depend little on the choice of $L$ \cite{Leimkuhler_2013, Margul_2016}.

\subsection{Multiple Time-Scale Integration}

The SIN(R) equations can be integrated using a single time-step size appropriate for the smallest relevant time scale in the system.
However, the most important feature of the method is related to its use in conjunction with multiple time-scale (MTS) integration.
In order to devise an MTS scheme using the reversible reference system propagator algorithm (RESPA) \cite{Tuckerman_1992}, we split the force on each DoF $\dof$ into $M$ terms as $F_\dof = \sum_{k=1}^M F_\dof^k$.
Here, the characteristic time scale of each force component increases with index $k$, meaning that $F_\dof^1$ is the fastest component while $F_\dof^M$ is the slowest one.
Then, we propose partitioning the Liouville operator in Eq.~\eqref{eq:isokinetic Liouville operator} as
\begin{equation}
\label{eq:RESPA Liouville Operator}
iL = iL_\mathrm{middle} + iL_\mathrm{move} + \sum_{k=1}^M \left( iL_\mathrm{force}^k + iL_\mathrm{side}^k \right).
\end{equation}

In this expression, $iL_\mathrm{move}$ is the only component that entails changes in the particle coordinates, while each $iL_\mathrm{force}^k$ is the only one that depends on the forces with superscript $k$.
Therefore, operators  $iL_\mathrm{middle}$ and $iL_\mathrm{side}^k$ are reserved to transformations in particle velocities due to the action of the thermostats, as well as to changes in the thermostat variables themselves.
The RESPA recipe then goes as follows.
At the largest time scale ($k=M$), integration is done by executing $n_M$ steps of size $\delta t_M = \Delta t$.
Internally, every step of size $\delta t_k$ taken at a time scale $k$ involves a number $n_{k-1}$ of substeps, each one of size $\delta t_k/n_{k-1}$.
Therefore, the step sizes at all scales and the total integration time $t$ are determined once we define $\Delta t$ and a set of numbers $\{n_k\}_{k=1}^M$.
The classical propagator representing our proposed RESPA scheme can be written in a recursive style as
\begin{subequations}
\label{eq:RESPA scheme}
\begin{equation}
e^{t iL} = \mathcal{G}_M(t),
\end{equation}
where an operator $\mathcal{G}_k(t)$ is defined as
\begin{multline}
\mathcal{G}_k(t) = \Big[e^{\frac{t}{2 n_k} iL_\mathrm{side}^k}
e^{\frac{t}{2 n_k} iL_\mathrm{force}^k}
\mathcal{G}_{k-1}\left(\tfrac{t}{n_k}\right) \times \\
\times e^{\frac{t}{2 n_k} iL_\mathrm{force}^k}
e^{\frac{t}{2 n_k} iL_\mathrm{side}^k}
\Big]^{n_k}.
\end{multline}

The definition above applies for $k \in [1, M]$.
The recursive process comes to an end when we make
\begin{equation}
\mathcal{G}_0(t) = e^{\frac{t}{2} iL_\mathrm{move}}
e^{t iL_\mathrm{middle}}
e^{\frac{t}{2} iL_\mathrm{move}}.
\end{equation}
\end{subequations}

The use of words \textit{middle} and \textit{side} was inspired by the work of Zhang \textit{et al}. \cite{Zhang_2017}, since Eq.~\eqref{eq:RESPA scheme} can be regarded as a MTS generalization of the unified thermostat method they discuss therein.

In order to fit Eq.~\eqref{eq:RESPA Liouville Operator} into Eq.~\eqref{eq:isokinetic Liouville operator}, we begin by realizing that
\begin{equation}
iL_\mathrm{move} = \sum_\dof v_\dof \diff{q_\dof}{},
\end{equation}
which makes the effect of a propagator $e^{t iL_\mathrm{move}}$ be a simple translation $q_\dof \leftarrow q_\dof + v_\dof t$ for all $\dof$.
Next, following Leimkuhler \textit{et al}. \cite{Leimkuhler_2013}, we define
\begin{equation}
\label{eq:force-dependent Liouville operator}
iL_\mathrm{force}^k = \sum_\dof \left[\left(\frac{F^k_\dof}{m_\dof} - \lambda^k_\dof v_\dof \right)\diff{v_\dof}{} - \lambda^k_\dof v_{1,\dof} \diff{v_{1,\dof}}{}\right],
\end{equation}
where the new Lagrange multiplier $\lambda^k_\dof$ ensures that the isokinetic constraint holds, as in Eq.~\eqref{eq:isokinetic constraint}, at an application of propagator $e^{t iL_\mathrm{force}^k}$ alone.
It is possible to compute the action of this propagator exactly \cite{Minary_2003, Leimkuhler_2013}, as we will discuss shortly.

Finally, the remaining components $iL_\mathrm{middle}$ and $\{iL_\mathrm{side}^k\}_{k=1}^M$ must be defined so that
\begin{multline}
iL_\mathrm{middle} + \sum_{k=1}^M iL_\mathrm{side}^k = \\
- \sum_{\dof=1}^{dN} \left[ \lambda^\ast_\dof v_\dof\diff{v_\dof}{} + \left(\lambda^\ast_\dof v_{1,\dof} + v_{2,\dof} v_{1,\dof}\right)\diff{v_{1,\dof}}{} \right] \\
+ \sum_{\dof=1}^{dN} \left[\tfrac{Q_1 v_{1,\dof}^2 - kT}{Q_2}\diff{v_{2,\dof}}{}
+ iL_{OU,\dof} \right],
\end{multline}
where $\lambda^\ast_\dof = \lambda_\dof - \sum_{k=1}^M \lambda^k_\dof$.
Evidently, there are multiple such definitions, which result in distinct integration procedures.
For instance, one can make $iL_\mathrm{middle} = \sum_{k=1}^M iL_{OU,\dof}$ and $iL_\mathrm{side}^k = 0$ for all but a selected index $k = k^\ast$.
In this way, the XO-RESPA scheme described in Ref.~\citenum{Leimkuhler_2013} corresponds to $k^\ast = M$.
Even though the XI-RESPA scheme \cite{Leimkuhler_2013} does not fit exactly into this notation, a close variant emerges by making $k^\ast = 1$.
Nevertheless, we introduce a new scheme here, based on the recent findings of Zhang and co-workers \cite{Zhang_2017}.
They verified that a thermostat tends to work better in terms of configurational sampling when its integration 


goes in-between coordinate updates.


This tendency had been noted already for Langevin thermostats \cite{Leimkuhler_2012, Leimkuhler_2013_2}, and the new study included other stochastic as well as deterministic thermostats.
% TODO: thorough reading of Leimkuhler_2012 and Leimkuhler_2013_2
In view of these findings, 


\section{Alternative RESPA Splitting}

The classical propagator representing our proposed RESPA scheme can be written in a recursive style as
\begin{subequations}
	\label{eq:RESPA scheme}
	\begin{equation}
	e^{t iL} = \mathcal{G}_M(t),
	\end{equation}
	where an operator $\mathcal{G}_k(t)$ is defined as
	\begin{multline}
	\mathcal{G}_k(t) = \Big[\mathcal{G}_{k-1}\left(\tfrac{t}{2n_k}\right)
	e^{\frac{t}{2 n_k} iL_\mathrm{force}^k}
	e^{\frac{t}{n_k} iL_\mathrm{middle}^k}
	 \times \\
	\times e^{\frac{t}{2 n_k} iL_\mathrm{force}^k}
	\mathcal{G}_{k-1}\left(\tfrac{t}{2n_k}\right)
	\Big]^{n_k}.
	\end{multline}
	
	The definition above applies for $k \in [1, M]$.
	The recursive process comes to an end when we make
	\begin{equation}
	\mathcal{G}_0(t) = e^{\frac{t}{2} iL_\mathrm{move}}
	e^{t iL_\mathrm{middle}}
	e^{\frac{t}{2} iL_\mathrm{move}}.
	\end{equation}
\end{subequations}

\subsection{Force-Dependent Isokinetic Propagator}

The equations of motion that stem from the Liouville operator in Eq.~\eqref{eq:force-dependent Liouville operator} are
\begin{subequations}
\label{eq:force-dependent isokinetic equations}
\begin{align}
& \dot{v}_\dof = \frac{F^k_\dof}{m_\dof} - \lambda^k_\dof v_\dof \quad \mathrm{and} \\
& \dot{v}_{1,\dof} = - \lambda^k_\dof v_{1,\dof}.
\end{align}
	
In this case, for Eq.~\eqref{eq:isokinetic constraint derivative} to be satisfied, it is required that
\begin{equation}
\lambda^k_\dof = \frac{F^k_\dof v_\dof}{m_\dof v_\dof^2 + \frac{1}{2} Q_1 v_{1,\dof}^2}.
\end{equation}
\end{subequations}

\subsection{Force-Independent Isokinetic Propagator}

\begin{equation}
\lambda^\ast_\dof = \frac{- \frac{1}{2} Q_1 v_{2,\dof} v_{1,\dof}^2}{m_\dof v_\dof^2 + \frac{1}{2} Q_1 v_{1,\dof}^2}.
\end{equation}


\section{DRAFT}




\begin{equation}
iL = iL_\mathrm{move} + iL_\mathrm{force} + iL_\mathrm{extra}.
\end{equation}

\begin{equation}
iL_\mathrm{move} = \sum_\dof v_\dof\diff{q_\dof}{}
\end{equation}

\begin{equation}
iL_\mathrm{force} = \sum_\dof \left[ \left(\frac{F_\dof}{m_\dof} - \lambda^\mathrm{F}_\dof v_\dof\right)\diff{v_\dof}{} - \lambda^\mathrm{F}_\dof v_{1,\dof}\diff{v_{1,\dof}}{} \right]
\end{equation}

\begin{equation}
iL_\mathrm{extra} = - \lambda^\ast_\dof v_\dof\diff{v_\dof}{} - \left(\lambda^\ast_\dof v_{1,\dof} + v_{2,\dof} v_{1,\dof}\right)\diff{v_{1,\dof}}{} \\
+ \tfrac{Q_1 v_{1,\dof}^2 - kT}{Q_2}\diff{v_{2,\dof}}{}
+ iL_{OU,\dof}
\end{equation}

\begin{multline}
iL_\dof = v_\dof\diff{q_\dof}{} \\
+ \sum_{k=1}^M \left[ \left(\frac{F^k_\dof}{m_\dof} - \lambda^k_\dof v_\dof\right)\diff{v_\dof}{} - \lambda^k_\dof v_{1,\dof}\diff{v_{1,\dof}}{} \right] \\
- \lambda^\ast_\dof v_\dof\diff{v_\dof}{} - \left(\lambda^\ast_\dof v_{1,\dof} + v_{2,\dof} v_{1,\dof}\right)\diff{v_{1,\dof}}{} \\
+ \tfrac{Q_1 v_{1,\dof}^2 - kT}{Q_2}\diff{v_{2,\dof}}{}
+ iL_{OU,\dof}
\end{multline}

\section{Results}

\subsection{Solvation Free Energy}
\label{sec:solvation free energy}


\section{Conclusion}


\begin{acknowledgement}

C.R.A.A. acknowledges the financial support of Petrobras (project code CENPES 16113).

\end{acknowledgement}

\bibliography{sinr}

\end{document}