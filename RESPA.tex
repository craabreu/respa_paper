\documentclass[
    journal=jctcce,
    layout=twocolumn
]{achemso}

\setkeys{acs}{
	abbreviations = false,
	articletitle  = false,
	keywords      = false,
	maxauthors    = 10,
	super         = true
}

% Comment below before submitting:
\let\titlefont\undefined
\usepackage[fontsize=11pt]{scrextend}
\usepackage[hidelinks,colorlinks,citecolor=blue]{hyperref}
%\flushbottom
% Up to this point

\usepackage{amsmath}
\usepackage{amssymb}
\usepackage[T1]{fontenc}
%
\usepackage[table]{xcolor}
\definecolor{lightgray}{gray}{0.85}
%
\usepackage{array}
\newcolumntype{L}{>{$}l<{$}}
\newcolumntype{C}{>{$}c<{$}}
\newcolumntype{R}{>{$}r<{$}}
%
\newcommand{\mt}[1]{\boldsymbol{\mathbf{#1}}}   % matrix symbol
\newcommand{\vt}[1]{\boldsymbol{\mathbf{#1}}}   % vector symbol
\newcommand{\tr}[1]{#1^\text{t}}                % transposition
\newcommand{\diff}[2]{\frac{\partial #2}{\partial #1}} % derivative
%\newcommand{\diff}[2]{\nabla_{#1}{#2}} % derivative
%\newcommand{\diff}[2]{\partial_{#1}{#2}} % derivative
\newcommand{\avg}[1]{\overline{#1}}             % average

\newcommand{\dof}{i}   % index for each degree of freedom
\newcommand{\Liu}{i\!L}

%\listfiles

\author{Charlles R. A. Abreu}
\email{abreu@eq.ufrj.br}
\affiliation{Chemical Engineering Department, Escola de Quimica, Universidade Federal do Rio de Janeiro, Rio de Janeiro, RJ 21941-909, Brazil}
\alsoaffiliation{Department of Chemistry, New York University, New York, New York 10003, USA}

\author{Mark E. Tuckerman}
\email{marktuckerman@nyu.edu}
\affiliation{Department of Chemistry, New York University, New York, New York 10003, USA}
\alsoaffiliation{Courant Institute of Mathematical Sciences, New York University, New York, New York 10012, USA}
\alsoaffiliation{NYU-ECNU Center for Computational Chemistry at NYU Shanghai, Shanghai 200062, China}


\title{Alternative Splitting Schemes for Multiple Time-Scale Integrators}

\abbreviations{i.i.d., MC, MD, CLT, OBM, MSE, FEP, BAR, WHAM, MBAR, MICS}

\keywords{Free Energy Computation, Reweighting, Multistate, Uncertainty Estimation}

\begin{document}

%\begin{tocentry}
%Graphical Abstract
%\end{tocentry}

%\tableofcontents

\begin{abstract}
To be included.
\end{abstract}

\section{Introduction}
\label{sec:introduction}

Problems in the evaluation of configurational properties such as the atomic virial have been reported \cite{Andoh_2017}.

The SIN(R) method \cite{Leimkuhler_2013}.

Isokinetic distributions coincides with a canonical distribution if we look at the coordinates only.

\section{Multiple Time-Scale Integration}

Consider a system of $N$ particles in $d$ dimensions, whose every configurational degree of freedom $\dof$ has coordinate $q_\dof$, momentum $p_\dof$, and associated mass $m_\dof$.
The system is subject to a potential field $U(q)$ and, therefore, the force acting on $\dof$ is $F_\dof = -\diff{q_\dof}{U}$.
In addition, a set of extended phase-space variables is included so that the original phase space can be dynamically sampled in accordance with some specified statistical ensemble.
If $x$ is a vector comprising all dynamical variables, we can represent the equations of motion which describe a flow through the extended phase space as
\begin{equation}
\label{eq:general equation of motion}
\dot{x} = \Liu x,
\end{equation}
where $\Liu = \xi(x) \cdot \nabla_x$ is a generalized (i.e. non-Hamiltonian) Liouville operator \cite{Tuckerman_1999, Tuckerman_2001, Tuckerman_2006}.

The solution of this equation is formally written as $x_t = e^{t\Liu}x_0$.
However, it is usually necessary to approximate the action of the classical propagator $e^{t\Liu}$ numerically.
This can be done by sequentially applying a split operator $\prod_j e^{\Delta t\Liu_j} \approx e^{\Delta \Liu}$, where $\Delta t = t/n_M$ and $\sum_j \Liu_j = \Liu$.
This means that $n_M$ is the number of integration steps and $\Delta t$ is the step size.
According to the Trotter theorem \cite{Trotter_1959}, equality holds when $\Delta t \to 0$.

In order to devise a multiple time-scale (MTS) integration scheme using the reversible reference system propagator algorithm (RESPA) \cite{Tuckerman_1992}, we split the force on each degree of freedom $\dof$ into a sum of $M$ terms, i.e. $F_\dof = \sum_{k=1}^M F_\dof^k$.
Here, the characteristic time scale of each force component increases with index $k$, meaning that $F_\dof^1$ is the fastest component while $F_\dof^M$ is the slowest one.
In the basic RESPA recipe, integration at the largest time scale ($k=M$) is done by executing $n_M$ steps of size $\delta t_M = \Delta t$.
Internally, every step of size $\delta t_k$, taken at a time scale $k$, involves $n_{k-1}$ substeps of size $\delta t_k/n_{k-1}$ each.
Therefore, the step sizes at all time scales are determined by defining the total integration time $t$ and a set of integer numbers $\{n_k\}_{k=1}^M$.

Here, we propose partitioning the extended-system Liouville operator in Eq.~\eqref{eq:general equation of motion} as
\begin{equation}
\label{eq:RESPA Liouville Operator}
\Liu = \Liu_\mathrm{move} + \sum_{k=1}^M \Liu_\mathrm{force}^k + \sum_{k=0}^M \Liu_\mathrm{ext}^k,
\end{equation}
where $\Liu_\mathrm{move}$ is the only component which entails changes in the particle coordinates.
Likewise, each component $\Liu_\mathrm{force}^k$ is the only one that depends on the forces labeled with superscript $k$.
Then, operators $\{\Liu_\mathrm{ext}^k\}_{k=0}^M$ are reserved to transformations in particle momenta that depend exclusively on the state of extended-space variables.
Finally, we note that transformations in these variables can occur due to the action of any component in Eq.~\eqref{eq:RESPA Liouville Operator}.

The classical propagator representing a possible RESPA scheme can be written in a recursive manner as
\begin{equation}
\label{eq:RESPA outermost propagator}
e^{t \Liu} = \mathcal{G}_M(t),
\end{equation}
where $\mathcal{G}_M(t)$ belongs to a family of operators whose each member $\mathcal{G}_k(t)$ is defined as
\begin{multline}
\label{eq:RESPA scheme 1}
\mathcal{G}_k(t) = \Big[e^{\frac{t}{2 n_k} \Liu_\mathrm{ext}^k}
e^{\frac{t}{2 n_k} \Liu_\mathrm{force}^k}
\mathcal{G}_{k-1}\left(\tfrac{t}{n_k}\right) \times \\
\times e^{\frac{t}{2 n_k} \Liu_\mathrm{force}^k}
e^{\frac{t}{2 n_k} \Liu_\mathrm{ext}^k}
\Big]^{n_k}.
\end{multline}

This definition applies for $k \in [1, M]$, and the recursive process comes to an end when we make
\begin{equation}
\label{eq:RESPA innermost propagator}
\mathcal{G}_0(t) = e^{\frac{t}{2} \Liu_\mathrm{move}}
e^{t \Liu_\mathrm{ext}^0}
e^{\frac{t}{2} \Liu_\mathrm{move}}.
\end{equation}

As an example, for a system at constant volume subject to the action of a set of thermostats, we would have
\begin{equation}
\Liu_\mathrm{move} = \sum_\dof v_\dof \diff{q_\dof}{},
\end{equation}
where $v_\dof = \frac{p_\dof}{m_\dof}$ is the velocity of degree of freedom $\dof$, and
\begin{equation}
\Liu_\mathrm{force}^k = \sum_\dof F_\dof^k \diff{p_\dof}{},
\end{equation}
meaning that the effects of propagators $e^{t \Liu_\mathrm{move}}$ and $e^{t \Liu_\mathrm{force}^k}$ are simple translations $q_\dof \leftarrow q_\dof + v_\dof t$ and $p_\dof \leftarrow p_\dof + F_\dof^k t$ for all $\dof$, respectively.
Then, if the Liouville operator regarding the thermostat set is $\Liu_\mathrm{T}$, we have freedom to allocate its components throughout the several time scales, with the requirement that $\sum_{k=0}^M \Liu_\mathrm{ext}^k = \Liu_\mathrm{T}$.

The described MTS scheme becomes closely related to the extended system RESPA algorithms introduced in Ref.~\citenum{Martyna_1996} if one makes $\Liu_\mathrm{ext}^k = 0$ for all but a selected index $k^\ast \geq 1$.
This implies that the whole thermostat is integrated at a unique time scale.
For instance, the XO-RESPA scheme \cite{Martyna_1996, Leimkuhler_2013} is reproduced by making $k^\ast = M$.
Even though the XI-RESPA scheme \cite{Martyna_1996, Leimkuhler_2013} does not fit exactly into the proposed notation, a close variant XI\textsuperscript{*}-RESPA emerges by making $k^\ast = 1$.

Due to the presence of $e^{t \Liu_\mathrm{ext}^0}$ in Eq.~\eqref{eq:RESPA innermost propagator}, the described scheme can also be regarded as a MTS generalization of the approach which Zhang and co-workers presented recently as a unified thermostat approach \cite{Zhang_2017}.
By making $\Liu_\mathrm{ext}^0 = \Liu_\mathrm{T}$ and $\Liu_\mathrm{ext}^k = 0$ for all $k > 0$, one reproduces the so-called \textit{middle} scheme \cite{Zhang_2017}, in which the thermostat integration occurs in-between coordinate moves.
In this way, thermostat-induced changes in particle momenta acquire a more direct influence on the resulting new coordinates.
As  Zhang \textit{et al}. \cite{Zhang_2017} observed, this tends to produce better coordinate sampling when compared to the more traditional \textit{side} schemes \cite{Zhang_2017}.
Such a tendency had been noted already for Langevin thermostats \cite{Leimkuhler_2012, Leimkuhler_2013_2}, but the new study extended the observation for other stochastic as well as deterministic thermostats.

Zhang \textit{et al}. \cite{Zhang_2017} also noted that the middle-scheme improvement in coordinate sampling is achieved at the cost of a degraded sampling of momenta and related properties.
It is implied here that configurational sampling occurs right after the end of a complete time step.
Hence, it is performed after a momentum move in both the middle and side schemes, with the difference that such move is force-induced in the former but thermostat-induced in the latter.
It is thus not surprising that the momenta sampled with the side scheme comply more closely with a canonical distribution than those sampled with the middle scheme.

TODO: prepare a figure comparing the different schemes. 

The authors also tested a third approach whose operator splitting is identical to the side scheme, thus producing exactly the same dynamics, 




momentum move in the middle scheme and after a thermostat-induced momentum move in the side scheme.


s, but with an important difference.
in the former such a momentum move comprises 

We argue, however, that this is a mere convention, since an equally valid integration method could be devised by shifting the endpoint of each step to some other place in the sequence of propagators.

We propose a different scheme by altering the order of the operators in Eq.~\eqref{eq:RESPA scheme 2}.
In this new scheme, $\mathcal{G}_k(t)$ becomes
\begin{multline}
\label{eq:RESPA scheme 2}
\mathcal{G}_k(t) = \Big[\mathcal{G}_{k-1}\left(\tfrac{t}{2n_k}\right)
e^{\frac{t}{2 n_k} \Liu_\mathrm{force}^k}
e^{\frac{t}{n_k} \Liu_\mathrm{ext}^k}
\times \\
\times e^{\frac{t}{2 n_k} \Liu_\mathrm{force}^k}
\mathcal{G}_{k-1}\left(\tfrac{t}{2n_k}\right)
\Big]^{n_k}.
\end{multline}

In this way, sampling is performed after an update in the smallest time scale.

\section{Stochastic Isokinetic Method}

\subsection{Equations of Motion}

In the SIN(R) method, every dof gets acted upon by an individual set of thermostats.
This is known as massive thermostatting.
For this, we define a pair of extended-system variables $v_{1,\dof}$ and $v_{2,\dof}$ for every dof $\dof$, with associated inertial parameters $Q_1$ and $Q_2$, respectively.
In practice, a single value $Q_1 = Q_2 = kT\tau^2$ is employed, where $k$ is the Boltzmann constant, $T$ is the temperature of the heat bath, and $\tau$ is a relevant time scale of the system dynamics.
The stochastic differential equation (SDE) system which prescribes such dynamics is
\begin{subequations}
\label{eq:equations of motion}
\begin{align}
& \dot{q}_\dof = v_\dof, \label{eq:q} \\
& \dot{v}_\dof = \frac{F_\dof}{m_\dof} - \lambda_\dof v_\dof, \label{eq:v} \\
& \dot{v}_{1,\dof} = - \lambda_\dof v_{1,\dof} - v_{2,\dof} v_{1,\dof}, \quad \mathrm{and}  \label{eq:v1} \\
& dv_{2,\dof} = \tfrac{Q_1 v_{1,\dof}^2 - kT}{Q_2}dt - \gamma v_{2,\dof} dt + \sqrt{\tfrac{2 \gamma kT}{Q_2}} dW_\dof, \label{eq:v2}
\end{align}
\end{subequations}
where $\gamma$ is a friction constant and $dW_\dof$ represents an infinitesimal increment of a stochastic Wiener process.
In addition, a Lagrange multiplier $\lambda_\dof$ is introduced with the aim of imposing an isokinetic constraint to each dof, which is \cite{Leimkuhler_2013, Margul_2016}
\begin{subequations}
\label{eq:isokinetic constraint}
\begin{equation}
m v_\dof^2 + \frac{1}{2} Q_1 v_{1,\dof}^2 = kT.
\end{equation}

Since this implies that
\begin{equation}
\label{eq:isokinetic constraint derivative}
m v_\dof \dot{v}_\dof + \frac{1}{2} Q_1 v_{1,\dof}\dot{v}_{1,\dof} = 0,
\end{equation}
\end{subequations}
substitution of $\dot{v}_\dof$ and $\dot{v}_{1,\dof}$ from Eqs.~\eqref{eq:v} and \eqref{eq:v1} leads to
\begin{equation}
\lambda_\dof = \frac{F_\dof v_\dof - \frac{1}{2} Q_1 v_{2,\dof} v_{1,\dof}^2}{m_\dof v_\dof^2 + \frac{1}{2} Q_1 v_{1,\dof}^2}.
\end{equation}

As usual, we can express a set of equations of motion as the effect of a Liouville operator over the dynamic variables of the system.
In the present case, such operator can be expressed as
\begin{multline}
\label{eq:isokinetic Liouville operator}
\Liu = \sum_{\dof=1}^{dN} \bigg[ v_\dof\diff{q_\dof}{} + \left(\frac{F_\dof}{m_\dof} - \lambda_\dof v_\dof\right)\diff{v_\dof}{} \\
- \left( \lambda_\dof v_{1,\dof} + v_{2,\dof} v_{1,\dof} \right) \diff{v_{1,\dof}}{} \\
+ \tfrac{Q_1 v_{1,\dof}^2 - kT}{Q_2}\diff{v_{2,\dof}}{}
+ \Liu_{OU,\dof} \bigg],
\end{multline}
where $\Liu_{OU,\dof}$ would, if isolated, represent a drift-free Ornstein-Uhlenbeck process involving variable $v_{2,\dof}$.
As demonstrated in Ref.~\citenum{Leimkuhler_2013}, the SIN(R) method dynamics is ergodic and preserves the isokinetic distribution.
The equations resemble a Nos\'{e}-Hoover-Langevin thermostat \cite{Samoletov_2007, Leimkuhler_2009}, with $v_{1,\dof}$ acting as intermediary between the system dof $\dof$ and a Langevin-type thermostat.
It is also possible to consider the simultaneous action of a set of $2L$ thermostat variables per dof \cite{Minary_2003, Minary_2003_2}.
Throughout this document, however, we only adopt the simplest case ($L=1$) because the method's efficacy has been verified to depend little on the choice of $L$ \cite{Leimkuhler_2013, Margul_2016}.

\subsection{Force-Dependent Isokinetic Propagator}

The equations of motion that stem from the Liouville operator in Eq.~\eqref{eq:force-dependent Liouville operator} are
\begin{subequations}
\label{eq:force-dependent isokinetic equations}
\begin{align}
& \dot{v}_\dof = \frac{F^k_\dof}{m_\dof} - \lambda^k_\dof v_\dof \quad \mathrm{and} \\
& \dot{v}_{1,\dof} = - \lambda^k_\dof v_{1,\dof}.
\end{align}
	
In this case, for Eq.~\eqref{eq:isokinetic constraint derivative} to be satisfied, it is required that
\begin{equation}
\lambda^k_\dof = \frac{F^k_\dof v_\dof}{m_\dof v_\dof^2 + \frac{1}{2} Q_1 v_{1,\dof}^2}.
\end{equation}
\end{subequations}

\subsection{Force-Independent Isokinetic Propagator}

\begin{equation}
\lambda^\ast_\dof = \frac{- \frac{1}{2} Q_1 v_{2,\dof} v_{1,\dof}^2}{m_\dof v_\dof^2 + \frac{1}{2} Q_1 v_{1,\dof}^2}.
\end{equation}


\section{DRAFT}




\begin{equation}
\Liu = \Liu_\mathrm{move} + \Liu_\mathrm{force} + \Liu_\mathrm{extra}.
\end{equation}

\begin{equation}
\Liu_\mathrm{move} = \sum_\dof v_\dof\diff{q_\dof}{}
\end{equation}

\begin{equation}
\Liu_\mathrm{force} = \sum_\dof \left[ \left(\frac{F_\dof}{m_\dof} - \lambda^\mathrm{F}_\dof v_\dof\right)\diff{v_\dof}{} - \lambda^\mathrm{F}_\dof v_{1,\dof}\diff{v_{1,\dof}}{} \right]
\end{equation}

\begin{equation}
\Liu_\mathrm{extra} = - \lambda^\ast_\dof v_\dof\diff{v_\dof}{} - \left(\lambda^\ast_\dof v_{1,\dof} + v_{2,\dof} v_{1,\dof}\right)\diff{v_{1,\dof}}{} \\
+ \tfrac{Q_1 v_{1,\dof}^2 - kT}{Q_2}\diff{v_{2,\dof}}{}
+ \Liu_{OU,\dof}
\end{equation}

\begin{multline}
\Liu_\dof = v_\dof\diff{q_\dof}{} \\
+ \sum_{k=1}^M \left[ \left(\frac{F^k_\dof}{m_\dof} - \lambda^k_\dof v_\dof\right)\diff{v_\dof}{} - \lambda^k_\dof v_{1,\dof}\diff{v_{1,\dof}}{} \right] \\
- \lambda^\ast_\dof v_\dof\diff{v_\dof}{} - \left(\lambda^\ast_\dof v_{1,\dof} + v_{2,\dof} v_{1,\dof}\right)\diff{v_{1,\dof}}{} \\
+ \tfrac{Q_1 v_{1,\dof}^2 - kT}{Q_2}\diff{v_{2,\dof}}{}
+ \Liu_{OU,\dof}
\end{multline}

\section{Results}

\subsection{Solvation Free Energy}
\label{sec:solvation free energy}


\section{Conclusion}


\begin{acknowledgement}

C.R.A.A. acknowledges the financial support of Petrobras (project code CENPES 16113).

\end{acknowledgement}

\bibliography{RESPA}

\end{document}